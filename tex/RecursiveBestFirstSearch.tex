\chapter{Recursive Best-First Search} 

The recursive best-first search, similarly to the depth-limited search (\ref{sec:dls}), searches for the deepest node first, but stops after surpassing a bound. But in this case the bound is placed on the f-value of a node, instead of its depth.

The f-value of a node is the maximum between the f-value of its parent and the cost of reaching the node + the heuristic, i.e. $f(n) = max\{ g(n) + h(n), f(n.parent) \}$. When a branch of the search is cut off because the limit on the f-value, while backtracking, the f-value of a parent node will be updated with the f-value of its child. This way, the f-value of a node will become a more and more accurate estimation of the true cost of reaching the goal node from it.

This algorithm uses the principle of best-first searching. It only expands the node with the smallest f-value. If the node is on the current searching branch, than it will be expanded, otherwise the search tree backtracks to the level of the node with the smallest f-value. Therefore at each level of recursion we take the first two successors with the lowest f-value, call the search for the best option, and supply the bound as the f-limit of the alternative successor. This is repeated until the solution is found, or until the f-value of the best node will become greater than the f-limit. Thus it always considers a best and an alternative path, and switches between them, if the alternative path becomes the best one, in terms of lowest f-value.

For the implementation the \verb|Node| class was used for storing the nodes, and a priority queue was used for obtaining the best successor of a node. The recursive function returns the result and None, if the goal node was found. If the goal was not found, it returns None and the f-value of the deepest node reached before surpassing the limit.

Source code: Appendix \ref{sec:code_rbfs}.

\begin{figure}[ht]
    \centering
    \includegraphics[width=.8\linewidth]{fig/rbfs.png}
    \caption{Finding a path from Arad to Bucharest by using recursive best-first search. The number above a node is its f-limit, and the number on the right of it is its f-value. The searching branch was switched two times. \cite{aima2020}} 
    \label{fig:rbfs}
\end{figure}

