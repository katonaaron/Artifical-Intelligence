\chapter{Source Code: Search}

The code snippets below are from \verb|search.py|.



\section{Nodes}
\label{sec:nodes}

To represent the search nodes, we used different strategies across the various search algorithms. In some cases we used tuples containing the state, the action and the cumulative cost of a node, while in other cases we used the \verb|Node| class defined below.

In the latter case, we stored in a node a reference to its parent. This information is used for reconstructing the path from the start node to the goal node, in the \verb|reconstructPath(node)| method.

\lstinputlisting[language=Python, firstnumber=67, caption=The Node class]{code/node.py}

\lstinputlisting[language=Python, firstnumber=87,caption=Method for reconstructing the path from the start to the goal node.]{code/reconstruct_path.py}
  
  
  
\section{Bidirectional Search}
\label{sec:code_bs}

\lstinputlisting[language=Python, firstnumber=228]{code/bidirectional_search.py}



\section{Depth Limited Search}
\label{sec:code_dls}

\lstinputlisting[language=Python, firstnumber=366]{code/depth_limited_search.py}



\section{Iterative Deepening Depth-First Search}
\label{sec:code_iddfs}.

\lstinputlisting[language=Python, firstnumber=416]{code/iterative_deepening_search.py}



\section{Iterative Deepening A*}

\lstinputlisting[language=Python, firstnumber=433]{code/iterative_deepening_astar.py}



\section{Recursive Best-First Search}
\label{sec:code_rbfs}

\lstinputlisting[language=Python, firstnumber=301]{code/recursive_best_first_search.py}

